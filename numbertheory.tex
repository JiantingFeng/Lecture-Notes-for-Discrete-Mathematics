\section{整数的整除性}
\begin{definition}
	给定两个整数$a$和$b$且$b\neq 0$,则存在唯一的整数满足
	\begin{equation*}
		a = qb+r\quad\text{and}\quad 0\leq r< |b|
	\end{equation*}
\end{definition}
\begin{definition}
	Divisor function$\sigma_x(n)$定义为
	\begin{equation*}
		\sigma_x(n) = \sum\limits_{d|n}d^x
	\end{equation*}
	其中$x\in \mathbb{R}$, $n\in \mathbb Z^+$
\end{definition}
\begin{definition}
	一个数论函数$f(n)$被称为积性(multiplicative)的如果满足
	\begin{equation*}
		gcd(n, n') = 1\Rightarrow f(nn') = f(n)f(n')
	\end{equation*}
\end{definition}
\begin{proposition}
	对于$a, b, c\in \mathbb Z$,我们有如下结论:
	\begin{enumerate}
		\item 若 $b|a$且$c|b$,则$c|a$
		\item 若$b|a$,则$bc|ac$
		\item 若$c|a$且$c|b$,则$c|(ma+nb)$对于任意$m, n\in \mathbb{Z}$.
	\end{enumerate}
\end{proposition}
\begin{proposition}
	如果$f(n)$是积性函数,那么下面的函数也是
	\begin{equation*}
		g(n) = \sum_{d|n}f(n).
	\end{equation*}
\end{proposition}
\begin{proof}
	假设$gcd(n, n')=1$,则
	\begin{equation*}
		g(nn') = \sum_{c|nn'}f(c) = \sum_{d|n, d'|n'}f(dd') = \sum_{d|n}f(d)\sum_{d'|n'}f(d') = g(n)g(n')
	\end{equation*}
\end{proof}
\begin{theorem}[裴蜀定理]
	令$a, b\in \mathbb{Z}\backslash\{0\}$,且有$d = gcd(a, b)$,则存在$x, y\in \mathbb{Z}$使得
	\begin{equation*}
		ax+by = d
	\end{equation*}
	进一步,形如$ax+by$的数都是$d$的倍数。
\end{theorem}
\begin{proposition}
	如果$ka\equiv kb(mod\ m)$且$gcd(k, m) = d$,则有
	\begin{equation*}
		a\equiv b (mod\ m/d)
	\end{equation*}
	进一步,若$gcd(k, m) = 1$,则有$a\equiv b (mod\ m)$
\end{proposition}
\begin{definition}
	对于任意$m\in\mathbb{Z}^+$记$\mathbb{Z}_m = \{0, 1,\cdots, m-1\}$为$m$的完全剩余系。用$\mathbb{Z}_m^*$表示$\mathbb{Z}_m$的子集满足其中的元素都与$m$互素。Euler totient function被定义为$\phi(m) = |\mathbb{Z}_m^*|$,称作$m$的完全剩余类, 容易知道$(\mathbb{Z}_m^*,\cdot)$构成一个群(满足结合律、单位元、逆元)。
\end{definition}
\begin{theorem}[中国剩余定理(Chinese remainder theorem)]
	令$m_1, \cdots, m_k$两两互素,令$a_1, \cdots, a_k$,则下面的同余方程组一定有解
	\begin{equation*}
		\begin{cases}
			x \equiv a_1\mod m_1\\
			x \equiv a_2\mod m_2\\ 
			\vdots \\
			x \equiv a_k\mod m_k
		\end{cases}
	\end{equation*}
	且任意两个解都关于$\prod_{j=1}^km_j$同余。
\end{theorem}
\begin{proof}
	取如下的$x$即可
	\begin{equation*}
		x = \sum\limits_{j=1}^k n_j M_j a_j
	\end{equation*}
	其中$M_j=\prod\limits_{i\neq j}m_i$,$n_jM_j\equiv 1\mod a_j$,即$n_j=M_j^{-1}$(即$M_j$在$\mathbb{Z}_{a_j}$中的逆元)
\end{proof}
\begin{definition}[本原单位根]
	令$n\in \mathbb{Z^+}$,复数$\xi\in \mathbb{C}$是$n$次单位根如果
	\begin{equation*}
		\xi^n = 1.
	\end{equation*}
	如果满足
	\begin{equation*}
		\xi^r \neq 1
	\end{equation*}
	对于任意$1\leq r\leq n-1$,称$\xi$是一个本原单位根。
\end{definition}
\begin{proposition}
	$n$次单位根为
	\begin{equation*}
		\xi_h = e^{2\pi ih/n},\quad h = 0,1,\cdots, n-1.
	\end{equation*}
进一步有,\begin{enumerate}
	\item $\xi_h$是$n$ 次本原根当且仅当$gcd(h, n) = 1$
	\item $\xi_h$是第$r$个$n$的某个因此次本原根
\end{enumerate}
\end{proposition}
\section{质数}
\begin{definition}
	一个整数$p$满足$p>1$且$p$没有除了自己本身和$1$两个因子之外的其他因子,则$p$称为质数(prime),否则称为合数(composite)。
\end{definition}
\begin{theorem}[欧几里得第一定理]
	如果$p$是一个质数且$p\mid ab$,则要么$p\mid a$或$p\mid b$。
\end{theorem}
\begin{proof}
	设$p\nmid a$,则由质数的定义$gcd(p, a) = 1$,由Bezout定理,存在$s, t\in \mathbb{Z}$使得
	\begin{equation*}
		ps + at = 1
	\end{equation*}
	同乘$b$,有
	\begin{equation*}
		psb + abt = b
	\end{equation*}
	由$p\mid ab$,所以$p\mid b$。
\end{proof}
\begin{theorem}
	任意$k$个连续的正整数都可以被$k!$除尽。
\end{theorem}
\begin{proof}
	由组合数的定义
	\begin{equation*}
		\frac{n(n-1)\cdots(n-k+1)}{k!} = \binom{n}{k}\in\mathbb{Z}
	\end{equation*}
	而上式左端即为$k$个连续的正整数的乘积,通过调整$n$的值即可证明结论。
\end{proof}
\begin{corollary}
	如果$p$是一个质数,则$p\mid \binom{p}{k}$对于任意$1\leq k\leq p-1$
\end{corollary}
\begin{proof}
	容易知道
	\begin{equation*}
		p\cdot \frac{(p-1)!}{k!(p-k)!} = \binom{p}{k} \in \mathbb{Z}
	\end{equation*}
	由于$p$是质数,故分母$k!(p-k)!$与$p$互素,所以一定有
	\begin{equation*}
		\frac{(p-1)!}{k!(p-k)!}\in\mathbb{Z}
	\end{equation*}
	因此,$p\mid \binom{p}{k}$。
\end{proof}
\begin{theorem}[欧几里得第二定理]
	质数的个数有无限个。
\end{theorem}
\begin{proof}
	反证法,假设素数有有限个$p_1, p_2, \cdots,p_k$,则
	\begin{equation*}
		p_1p_2\cdots p_k +1
	\end{equation*}
	一定有素因子$p_i$,这是不可能的。
\end{proof}
\begin{theorem}[算数基本定理(The fundamental theorem of arithmetic)]
	任意一个大于$1$的数要么是素数,要么可以被唯一分解为素因子的乘积。
\end{theorem}
\begin{theorem}
	令$a,b\in \mathbb{Z}^+$,若
	\begin{equation*}
		a = p_1^{\alpha_1}p_2^{\alpha_2}\cdots p_m^{\alpha_m}\quad\text{与}\quad
		b = p_1^{\beta_1}p_2^{\beta_2}\cdots p_m^{\beta_m}
	\end{equation*}
	则有
	\begin{equation*}
		gcd(a, b) = \prod\limits_{i=1}^mp_i^{\min (\alpha_i,\beta_i)}
	\end{equation*}
	\begin{equation*}
		lcm(a, b) = \prod\limits_{i=1}^mp_i^{\max (\alpha_i,\beta_i)}
	\end{equation*}
	并且有$gcd(a,b)\cdot lcm(a, b) = ab$
\end{theorem}
\begin{note}
	事实上,这给出了最大公约数与最小公倍数的一种表示方法。
\end{note}
\begin{theorem}
	\begin{enumerate}
		\item 有无穷的$4n+3$型质数;
		\item 有无穷的$6n+5$型质数。
	\end{enumerate}
\end{theorem}
\begin{proof}
	只证1,若只有有限多个$4n+3$型质数,设最大的为$p$,则
	\begin{equation*}
		2^2\cdot 3 \cdot 5 \cdot \cdots \cdot p - 1
	\end{equation*}
	也是$4n+3$型质数,同时大于$p$,矛盾。
\end{proof}
\begin{theorem}[Dirichlet定理]
	设$a$是正整数而且$a,b$没有除$1$外其他因子,则有无限多个$an+b$型质数。
\end{theorem}
\begin{theorem}[素数定理]
	记$\pi(x)$为不超过$x$的素数的个数,则
	\begin{equation*}
		\pi(x)\sim \frac{x}{\log x}
	\end{equation*}
	换句话说,第$n$个质数$p_n$满足
	\begin{equation*}
		p_n\sim n\log n
	\end{equation*}
\end{theorem}
\section{Euler totient函数与Mobius反演公式}
\begin{definition}[Mobius函数]
	一个整数$n$被称为是一个square如果存在整数$k$满足$n = k^2$,cube的定义类似。Mobius函数$\mu(n)$被定义为
	\begin{enumerate}
		\item $\mu(1) = 1$
		\item $\mu(n) = 0$如果$n$是一个square
		\item $\mu(p_1p_2\cdots p_k) = (-1)^k$,如果$p_1, p_2, \cdots, p_k$是不同的质数
	\end{enumerate}
\end{definition}
\begin{example}
	\begin{equation*}
		\mu(2) = -1,\mu(4) = 0,\mu(6) = 1
	\end{equation*}
\end{example}
下面的定理给出了$Mobius$函数的一些性质
\begin{proposition}
	\begin{enumerate}
		\item 对于任意$n\in \mathbb{Z}^+$,
		\begin{equation*}
			\sum\limits_{d\mid n}\mu(d) = 
			\begin{cases}
				1,\quad \text{如果}n=1;\\
				0,\quad \text{如果}n\geq 2
			\end{cases}
		\end{equation*}
		\item 对于任意$n\in \mathbb{Z}^+$,
		\begin{equation*}
			\sum\limits_{d\mid n}|\mu(d)| = 2^k
		\end{equation*}
		其中$k$为$n$的素因子个数
	\end{enumerate}
\end{proposition}
\begin{theorem}[Mobius反演公式]
	\begin{equation*}
		g(n) = \sum\limits_{d\mid n}f(d) \Leftrightarrow f(n) = \sum\limits_{d\mid n}\mu(d)g(n/d)
	\end{equation*}
\end{theorem}
\begin{proof}
	从左到右
	\begin{equation*}
		\sum\limits_{d\mid n}\mu(d)g(n/d) = \sum\limits_{d\mid n}\mu(d)\sum\limits_{c\mid (n/d)}f(c) = \sum\limits_{cd\mid n}\mu(d)f(c) = \sum\limits_{c|n}f(c)\sum\limits_{d|(n/c)}\mu(d) = f(n)
	\end{equation*}
	从右向左
	\begin{equation*}
		\sum\limits_{d\mid n}f(d) =\sum\limits_{d\mid n} \sum\limits_{c\mid d} \mu(c)g(d/c) = \sum\limits_{d'\mid n}g(d')\sum\limits_{c\mid (n/d')}\mu(c) = g(n)
	\end{equation*}
\end{proof}
\begin{theorem}
	假设$gcd(m, m')=1$,则
	\begin{equation*}
		\begin{split}
			\mathbb{Z}_{mm'} &= \{am'+a'm:a\in\mathbb{Z}_m,a'\in\mathbb{Z}_{m'}\}\\
			\mathbb{Z}_{mm'}^* &= \{am'+a'm:a\in\mathbb{Z}_m^*,a'\in\mathbb{Z}_{m'}^*\}
		\end{split}
	\end{equation*}
\end{theorem}
\begin{proposition}
	下面的这些函数是积性函数
	\begin{enumerate}
		\item 最大公约数函数 $gcd(n, k)$对任意给定的整数$k$
		\item divisor function $\sigma_k(n)$对于任意$k\in\mathbb{N}$
		\item Mobius函数$\mu(n)$
		\item Euler totient函数$\phi(n)$.
	\end{enumerate}
\end{proposition}
\begin{theorem}
	设$n\in \mathbb{Z}^+$,则
	\begin{equation*}
		\phi(n) = n\prod\limits_{\text{prime }p\mid n}\big(1-\frac{1}{p}\big) = \sum\limits_{d\mid n}d\cdot \mu(\frac{n}{d})
	\end{equation*}
\end{theorem}
\begin{theorem}
	设$n\in \mathbb{Z}^+$,则
	\begin{equation*}
		\sum\limits_{d\mid n}\phi(d) = n
	\end{equation*}
\end{theorem}
\begin{note}
	上式实际上说明了$\phi(n)$	的Mobius反演为$n$。
\end{note}
\begin{theorem}[Fermat小定理]
	设$a\in \mathbb{N}$,若$p$是质数,则有
	\begin{equation*}
		a^p \equiv a\mod p
	\end{equation*}
	或者
	\begin{equation*}
		a^{p-1}\equiv 1\mod p
	\end{equation*}
\end{theorem}
\begin{theorem}[Fermat-Euler定理]
	若$gcd(a, m) = 1$,则
	\begin{equation*}
		a^{\phi(m)} \equiv 1\mod m.
	\end{equation*}
\end{theorem}
\section{二次剩余}
\begin{definition}
	设$m\in \mathbb{Z}^+$, $a$为模$m$的剩余,我们称$a$是二次剩余(quadratic residue)如果下面的同余方程有解
	\begin{equation*}
		x^2 \equiv a \mod m
	\end{equation*}
	,否则称为二次非剩余(quadratic nonresidue)。对于一个奇质数$p$和整数$m$,定义Legendre符号为
	\begin{equation*}
		\Big(\frac{a}{p}\Big) = \begin{cases}
			1,\quad \text{如果$a$是$p$的二次剩余且$p\nmid a$}\\
			-1,\quad \text{如果$a$是$p$的二次非剩余}\\
			0,\quad\text{如果$p\mid a$}
		\end{cases}
	\end{equation*}
	我们可以将Legendre符号推广到Jacobi符号
	\begin{equation*}
		\Big(\frac{a}{n}\Big) = \prod\limits_{j=1}^k\Big(\frac{a}{p_j}\Big)^{\alpha_j}
	\end{equation*}
	其中$n = p_1^{\alpha_1}p_2^{\alpha_2}\cdots p_k^{\alpha_k}$是$n$的素因子分解,特别地
	\begin{equation*}
		\Big(\frac{n}{1}\Big) = 1.
	\end{equation*}
\end{definition}
\begin{proposition}
	Jacobi符号满足下列的性质
	\begin{enumerate}
		\item 当$n = p$时, Jacobi符号就是Legendre符号
		\item 对于任意整数$m$,都有
		\begin{equation*}
			\Big(\frac{a}{n}\Big) = \Big(\frac{a+mn}{n}\Big) 
		\end{equation*}
		\item Jacobi符号等于零当且仅当$gcd(a, n)\neq 1$
		\item Jacobi符号对于每一个变量都是积性函数,i.e.
		\begin{equation*}
			\Big(\frac{ab}{n}\Big) = \Big(\frac{a}{n}\Big)\Big(\frac{b}{n}\Big)\quad\text{和}\quad
			\Big(\frac{a}{mn}\Big) = \Big(\frac{a}{m}\Big)\Big(\frac{a}{n}\Big)
		\end{equation*}
	\end{enumerate}
\end{proposition}
\begin{theorem}[Wilson定理]
	对于任意素数$p$,有
	\begin{equation*}
		(p-1)!\equiv -1\mod p
	\end{equation*}
\end{theorem}
\begin{theorem}[Euler判别法]
	若$p$是奇质数且$p$不整除$a$,则
	\begin{equation*}
		a^{\frac{p-1}{2}} \equiv\Big(\frac{a}{p}\Big) \mod p
	\end{equation*}
	其中是Legendre符号
\end{theorem}
\begin{example}
	对于给定数,寻找其为二次剩余的模数。
	如令$a = 17$,则$p=3$,有$17^{\frac{3-1}{2}}\equiv 2\mod 3\equiv -1\mod 3$,于是$17$不是$3$的二次剩余。
\end{example}
\begin{theorem}[Gauss引理]
	若$p$是奇质数且$p$不整除$a$,则Legendre符号
	\begin{equation*}
		\Big(\frac{a}{p}\Big) = (-1)^\mu
	\end{equation*}
	其中$\mu$是下列集合中数模去$p$大于$p/2$的数的个数
	\begin{equation*}
		S_p'(a) = \{a, 2a,\cdots, \frac{p-1}{2}\cdot a\}
	\end{equation*}
\end{theorem}
\begin{theorem}
	设$p$是一个奇素数,则共有$(p-1)/2$个二次剩余和$(p-1)/2$个二次非剩余,并且满足如下的关系
	\begin{enumerate}
		\item 两个二次剩余的乘积仍是二次剩余
		\item 两个二次剩余的乘积是二次非剩余
		\item 二次剩余和二次非剩余的乘积是二次非剩余
	\end{enumerate}
\end{theorem}
\begin{example}
	设$p = 11$和$a = 7$,则
	$S_{11}'(7) = \{7, 14,21, 28, 35\}$,模掉$11$后,大于$11/2$的个数$\mu = 3$,于是$\Big(\frac{7}{11}\Big) = -1$,即$7$不是$11$的二次剩余。
\end{example}
\begin{proposition}
	\begin{enumerate}
		\item $-1$是$4n+1$型质数的二次剩余,是$4n+3$型质数的二次非剩余
		\item $2$是$8n\pm 1$型质数的二次剩余,是$8n\pm 3$型质数的二次非剩余
		\item $-3$是$6n+1$型质数的二次剩余,是$6n+5$型质数的二次非剩余
		\item $7$是$10n\pm 1$型质数的二次剩余,是$10n\pm 3$型质数的二次非剩余
	\end{enumerate}
\end{proposition}
\begin{theorem}[Gauss二次互反律]
	\begin{equation*}
		\Big(\frac{p}{q} \Big)\Big(\frac{q}{p}\Big)=(-1)^{\frac{p-1}{2}\frac{q-1}{2}}
	\end{equation*}
\end{theorem}
\begin{example}
	\begin{equation*}
		\begin{split}
			\Big(\frac{1001}{9907}\Big) &= \Big(\frac{9907}{1001}\Big) = \Big(\frac{898}{1001}\Big) = \Big(\frac{2}{1001}\Big)\Big(\frac{449}{1001}\Big) = \Big(\frac{449}{1001}\Big) = \Big(\frac{1001}{449}\Big) = \Big(\frac{103}{449}\Big) = \Big(\frac{449}{103}\Big)\\
			&
=\Big(\frac{37}{103}\Big)= \Big(\frac{103}{37}\Big) = \Big(\frac{29}{37}\Big) = \Big(\frac{37}{29}\Big) = \Big(\frac{8}{29}\Big) = \Big(\frac{2}{29}\Big)^3 = -1
		\end{split}
	\end{equation*} 
\end{example}
\begin{note}
	注意到$29$是$8n\pm 3$型奇素数
\end{note}
\begin{example}
	\begin{equation*}
		\Big(\frac{37}{89}\Big) = \Big(\frac{89}{37}\Big) = \Big(\frac{15}{37}\Big) = \Big(\frac{5}{37}\Big)\Big(\frac{3}{37}\Big) = \Big(\frac{37}{5}\Big)\Big(\frac{37}{3}\Big) = \Big(\frac{2}{5}\Big)
		\Big(\frac{1}{3}\Big) = -1
		\end{equation*}
\end{example}