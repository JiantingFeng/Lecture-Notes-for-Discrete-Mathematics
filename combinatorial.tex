\section{二项式系数}
\begin{theorem}[二项式定理(Binomial theorem)]
	令$n\geq 1$为正整数。那么有
	\begin{equation*}
		(x+y)^n = \sum\limits_k \binom{n}{k}x^ky^{n-k}
	\end{equation*}
\end{theorem}
\begin{proposition}\label{combinatorial identities}
	二项式系数满足下列恒等式
	\begin{equation*}
		\binom{n}{k} = \binom{n}{n-k},
	\end{equation*}
	\begin{equation*}
		\binom{n}{k} = \binom{n-1}{k} + \binom{n-1}{k-1},
	\end{equation*}
	\begin{equation*}
		\binom{n}{k} = \frac{n}{k}\binom{n-1}{k-1},
	\end{equation*}
	\begin{equation*}
		\binom{a}{b}\binom{b}{c} = \binom{a}{c}\binom{a-c}{b-c},
	\end{equation*}
	\begin{equation*}
		\sum_k \binom{n}{k} = 2^k,
	\end{equation*}
	\begin{equation*}
		\sum_j \binom{n}{j}\binom{m}{k-j} = \binom{n+m}{k},
	\end{equation*}
	\begin{equation*}
		\sum_j \binom{n}{j}^2 = \binom{2n}{n},
	\end{equation*}
	\begin{equation*}
		\sum_j (-1)^j\binom{n}{j} = 0,
	\end{equation*}
	\begin{equation*}
		\sum_j \binom{n}{j}2^j = 3^n,
	\end{equation*}
	\begin{equation*}
		\sum_k \binom{n}{k}\binom{k}{m} = \binom{n}{m} 2^{n-m}.
	\end{equation*}
\end{proposition}
\begin{proposition}[Hockey-stick identity]
	\begin{equation*}
		\binom{n}{r} + \binom{n-1}{r}+\cdots + \binom{r}{r} = \binom{n+1}{r+1}
	\end{equation*}
\end{proposition}
\begin{proof}
	用两种方法计数(counting by two ways),其中一边是考虑固定最大的元素即可。
\end{proof}
\begin{proposition}
	对于任意$0\leq k\leq n$,
	\begin{equation*}
		\sum\limits_{j=0}^k (-1)^j \binom{n}{j} = (-1)^k\binom{n-1}{k}.
	\end{equation*}
\end{proposition}
\begin{proof}
	使用telescoping method,注意使用\ref{combinatorial identities}中的第二个式子即可。
\end{proof}
\begin{theorem}[Multinomial theorem]
	设$n\in \mathbb{N}$,则
	\begin{equation*}
		\big(\sum\limits_{j=1}^mx_j\big)^n = \sum\limits_{k_1+k_2+\cdots+k_m=n\newline k_1, \cdots,\_m\in \mathbb{N}}\binom{n}{k_1, k_2,\cdots, k_m}\prod\limits_{j=1}^m x_j^{k_j}
	\end{equation*}
\end{theorem}
\begin{proposition}
	将$n$个完全相同的物品分给$k$个人的分法数为
	\begin{equation*}
		\big(\tbinom{n+1}{k-1}\big)
	\end{equation*}
	若要求每个人都分到一件,则分法数为
	\begin{equation*}
		\binom{n+1}{k-1}
	\end{equation*}
\end{proposition}
\begin{proof}
	使用method of stars and bars
	\begin{equation*}
		\bigstar \bigstar \bigstar \bigstar \bigstar \bigstar \bigstar
	\end{equation*}
	\begin{equation*}
		\bigstar \bigstar \bigstar \bigstar \bracevert \bigstar \bracevert \bigstar \bigstar
	\end{equation*}
	\begin{equation*}
		\bigstar \bigstar \bigstar \bigstar \bracevert \bracevert \bigstar \bigstar \bigstar
	\end{equation*}
	注意在$n+1$个位置能放置$k-1$个bar即可。
\end{proof}
\begin{theorem}
	对于任意$n\in\mathbb Z$和$k\in \mathbb N$,有
	\begin{equation*}
		\big(\tbinom{n}{k}\big) = \binom{n+k-1}{k}
	\end{equation*}
\end{theorem}
\section{Catalan数}
\begin{definition}
	第$n$个Catalan数为
	\begin{equation*}
		C_n = \frac{1}{n+1}\binom{2n}{n}
	\end{equation*}
\end{definition}
Catalan数满足如下的递推公式,对于任意$n\geq 0$,
\begin{equation*}
	C_{n+1} = \sum\limits_{i=0}^nC_iC_{n-i}
\end{equation*}
通过递推公式容易得到Catalan数的生成函数
\begin{equation*}
	C(z) = \frac{1-\sqrt{1-4z}}{2z}
\end{equation*}
\begin{theorem}
	Catalan数$C_n$是以下结构的数量:
	\begin{enumerate}
		\item 凸$n+2$边形的三角剖分数量;
		\item $n$个顶点的二叉树;
		\item $n+1$个顶点的平面树;
		\item $2n$长的投票序列(Ballot sequences);
		\item 对$n+1$个$x$加括号(Parenthesizations or bracketings);
		\item 长为$2n$的Dyck paths的数量。
	\end{enumerate}
\end{theorem}
\section{Bell数}
\begin{definition}
	第$n$个Bell数$B_n$为集合$[n]$的拆分总数,即
	\begin{equation*}
		B_n = \sum\limits_{k=0}^n S(n, k)
	\end{equation*}
\end{definition}