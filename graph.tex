\section{基本定义与概念}
\begin{definition}[图]
	一个图(graph)是一个二元有序对$(V, E)$,其中$V$是所有顶点(vertices)的集合, $E$是所有边$edges$的集合。一条边$(u, v)$常被写作$uv$。$V(G)$和$E(G)$分别为图$G$的顶点和边的集合。一个图$G$的阶(order)是图的顶点个数(cardinality)$|V(G)|$。对于$v\in V(G)$,如果存在$uv\in E(G)$,我们称$u$和$v$是邻接(adjacent)的,否则称为非邻接(nonadjacent)的。顶点$v$的邻居(neighborhood),被记做$N_G(v)$是所有与$v$邻接的顶点的集合。
\end{definition}
\begin{definition}[简单图]
	一个简单图是一个既没有重边(multiple edge)也没有环(loop)的图。
\end{definition}
\begin{definition}[补图]
	给定一个简单图$G = (V, E)$,它的补图(complement)的定义为简单图
	\begin{equation*}
		\bar{G} = (V, \bar{E}),
	\end{equation*}
	其中$\bar{E} = V^2\backslash E$。意思是$\bar{G}$中的点之间有边相连当且仅当在$G$里面他们没有边相连。我们构建补图时可先构建完全图$K_{|V(G)|}$,再删除$G$中出现过的边即可。
\end{definition}
\begin{definition}[团]
	团(clique)是$G$中的一组满足两两之间有边连接的顶点的集合。图$G$的团数(clique number)是指图$G$中最大的团的阶数。
\end{definition}
\begin{definition}[独立集]
	图$G$独立集(independent set)或稳定集(stable set)是一族两点之间互不相连的定点的集合。$G$的独立数(independent number),记为$\alpha(G)$,是$G$中最大的独立集的阶数(order)。
\end{definition}
\begin{definition}[孤立点]
	一个顶点被称为孤立点当且仅当它没有邻居, i.e.
	\begin{equation*}
		N_G(v) = \emptyset
	\end{equation*}
\end{definition}
\begin{definition}[邻接矩阵与关联矩阵]
	图$G$是一个顶点和边分别为
	\begin{equation*}
		V = \{v_1, v_2, \cdots, v_n\} \text{ 和 } E = \{e_1, e_2, \cdots, e_m\}
	\end{equation*}
	图$G$的邻接矩阵(adjacency matrix)是指矩阵$(a_{ij})_{n\times n}$, 其中$a_{ij}$是顶点$v_i$和$v_j$间的边数(简单图中只能为0或1)。图$G$的关联矩阵(incidence matrix)是指矩阵$(b_{ij})_{n\times m}$,其中
	\begin{equation*}
		b_{ij} = \begin{cases}
			1, \text{如果顶点$v_i$与边$e_j$相连}\\
			0, \text{其他情况}
		\end{cases}
	\end{equation*}
	一个图的谱(spectrum)是指邻接矩阵的所有特征值的集合。
\end{definition}
\begin{definition}[度数]
	图$G$的一个顶点$v$的度数(degree)是与$v$连接的边的数量,常被记做$deg_G(v)$。度序列(degree sequence)是指一个由所有顶点的度从大到小排列而成的序列。一个图被称为正则的(regular)如果所有顶点的度相同。一个图被称为$k$-正则($k$-regular)如果它的所有顶点的度数都相同且为$k$。一个图被称为cubic的如果它是3-正则的。图$G$的最大与最小度被分别记为$\Delta(G)$与$\delta(G)$。如果一个顶点的度为$1$则它被称为叶(leaf)。一个连接叶的边被称为叶边(leaf edge)。
\end{definition}
\begin{note}
	一个环(loop)对度的贡献为$2$。
\end{note}
\begin{theorem}[握手引理(the Handshaking Lemma)]
	对于任何有限图$G = (V, E)$,我们有
	\begin{equation*}
		\sum\limits_{v\in V} deg\ v = 2|E|
	\end{equation*}
\end{theorem}
\begin{note}
	很自然的能看出图$G$中一定有偶数个度数为奇数的顶点。
\end{note}
\begin{theorem}
	一个非降的非负整数序列$d_1\geq d_2\geq \cdots \geq d_n$成为一个度序列的充分必要条件为$\sum\limits_{i=1}^n d_i$是偶数。
\end{theorem}
\begin{definition}
	设$G = (V, E)$是一个简单图。$G$的子图(subgraph)是图$(V', E')$,其中$E'\subseteq E$,$V'\subseteq V$,并且$V'$包含$E'$中所有边的端点。生成子图(spanning subgraph)也称作$G$的一个因子(factor),是一个子图并且包含原图$G$的所有端点。$G$有$k$-因子($k$-factor)是指一个每个顶点的度都为$k$的生成子图。图$G$的分解(factorization)是指将一个图的所有边都分拆为因子(factors)。诱导子图(induced subgraph)是子图$(V', G')$,如果$V'$中的顶点在$G$中相连,则他们一定在$G'$中相连。一个顶点集$U\subseteq V$的邻居(neighborhood)定义为$U$中所有顶点的邻居的并,即
	\begin{equation*}
		N_G(U) = \bigcup\limits_{v\in U}N_G(v)
	\end{equation*}
\end{definition}
\begin{definition}
	一个walk是一个边(edges)的序列。walk的长度是指它所含的边的个数。一个walk是闭合的(closed)当且仅当它的起始点和终点相同。一个trail是一个walk且不经过重复的边。一个path是一个trail且不经过重复的顶点。两个顶点的距离被定义为两个顶点间最短的path的长度。一张图的直径(diameter)是最短距离的最大值。图的Eulerian path是一个trail而且访问过所有边,一个Eulerian circuit是一条到访过所有边的trail。一个图被称为Eulerian如果它包含一个Eulerian 回路。Hamiltonian path是一个生成子图(到访所有顶点)。
\end{definition}
\begin{definition}
	一个$k$-cycle是简单图$G$中的闭合walk$v_1v_2\cdots v_kv_1$并且不经过重复的边,进一步,一个$k$-cycle同构与$C_k$。一个图被称为acyclic如果它不包含cycle,这样的图被称为森林(forest)。图的围长(girth)是指图中最短的圈常,circumference是指图中最长的圈的长度。一个图是Hamiltonian的当且仅当它的circumference等于它的度数(order)。
\end{definition}
\begin{definition}
	一个图被称为连通的(connected)如果每一对顶点之间都有一条path,否则被称为不连通的(disconnected)。一个连通分支(connected component)是指图$G$中最大的连通子图,一个奇分支(odd component)是一个阶为奇数的分支,否则称为偶分支。
\end{definition}
\begin{definition}
	对于$k\in \mathbb Z^+$,一个图被称为$k$-connected的如果去除任意$k-1$个顶点后该图都是连通的。图$G$连通度(connectivity),记做$\kappa (G)$是最大的$k$使得图$G$是$k$-connected。
\end{definition}
\begin{theorem}
	一个简单图$G$是连通的如果
	\begin{equation*}
		\delta(G)\geq \frac{|V(G)|}{2}.
	\end{equation*}
\end{theorem}
\begin{definition}
	令$k\geq 2$。一个$k$部图($k$-partite)是指图$G = (V, E)$满足它可以被分解为$k$个非空稳定集(stable sets)(内部不连通)。一个图被称为多部图(multipartite)如果$k\geq 2$。一个$2$-partite图也被称为bipartite图。它的顶点集可被分解为
	\begin{equation*}
		V = V_1\sqcup V_2
	\end{equation*}
\end{definition}
\begin{theorem}
	任意的连通二部图的分解是惟一的。
\end{theorem}
\begin{theorem}
	一个有限连通图是欧拉图当且仅当所有顶点的度数都是偶数。
\end{theorem}
\begin{definition}
	如下是一些常见图的例子
	\begin{enumerate}
		\item 空图(empty graph)是指图$(V, E)$满足$V = E = \emptyset$
		\item $n$阶完全图(complete graph),记做$K_n$是指有$n$个顶点且任意两个顶点都相邻的图。
		\item 完全$k$-部图(complete k-partite graph),记做$K_{n_1, n_2, \cdots, n_K} = (V, E)$是指简单图其顶点被分为了非空集合$V_1, V_2, \cdots , V_k$,其中$k\geq 2$且$|V_i|=n_i$对于任意$i$成立且满足
		\item \begin{equation*}
			E = \bigcup\limits_{i\neq j}(V_i\times V_j).
		\end{equation*}
		意思是任意两个不属于同一部分的顶点都有边相连。\par
		完全多部图是指完全$k$-部图且$k\geq 2$。一个完全二部图形为$K_{n_1, n_2}$。Turan图是一个完全$k$-部图$K_{n_1, n_2, \cdots, n_K}$且满足
		\begin{equation*}
			|n_i - n_j|\leq 1
		\end{equation*}
		对于任意$i$和$j$都成立。
		\item $n$阶path graph,记做$P_n$,是指一个简单图$(V, E)$,其中
		\begin{equation*}
			\begin{split}
				V &= \{v_1, v_2,\cdots, v_n\}\text{ 并且 }\\
				E &= \{v_1v_2, v_2v_3, \cdots, v_{n-1}v_n\}
			\end{split}
		\end{equation*}
		\item $n$阶cycle graph,记做$C_n$,是指简单图$(V, E)$,其中
		\begin{equation*}
			\begin{split}
				V &=\{v_1, v_2,\cdots, v_n\}\text{ 并且 }\\
				E &= \{v_1v_2, v_2v_3, \cdots, v_{n-1}v_n, v_nv_1\}
			\end{split}
		\end{equation*}
		$C_3, C_4, C_5, C_6$分别被称为 triangle, rectangle, pentagon和 hexagon。
		\item 树(tree)是一个不含cycle的连通图。星(star)是一个形如$K_1, n$的图,其中$k\geq 2$。 claw 是 star $K_1, 3$,caterpillar是一个树且满足它们所有顶点距离中心path的距离小于$1$,lobster是一个树且满足它们所有顶点距离中心path的距离小于$2$。
		\item 一个$n$阶($n\geq 4$)轮图(wheel graph),记做$W_n$,是一个简单图$(V, E)$满足
		\begin{equation*}
			\begin{split}
				V &= \{v_1, v_2,\cdots, v_n\}\text{ 并且 }\\
				E &= \{v_0v_i: 1\leq i \leq n-1\}\cup \{v_1v_2, v_2v_3,\cdots, v_{n-1}v_n, v_nv_1\}.
			\end{split}
		\end{equation*}
		\item 一个通过$n$阶完全图$K_n$删去一个顶点的图常被记做$K_n - e$。钻石(diamond graph)是指图$K_4 - e$
		\item Petersen graph是指简单图$G = (V, E)$满足
		\begin{equation*}
			V = \binom{[5]}{2}\text{ 和 }E = \{uv\in V^2: u\cap v =\emptyset\}
		\end{equation*}
		它有$10$个顶点$15$条边。
		\item 超立方图(hypercube graph),记做$Q_k$是一个简单图,顶点集合是一个$k$维$01$向量,两个顶点相邻当且仅当只有一个位置上的数不同。
	\end{enumerate}
\end{definition}
\begin{proposition}
	令$G$是一个简单图,$G$是一个完全多部图当且仅当$G$是$K_1+K_2$-free的
\end{proposition}
\begin{center}
	\begin{tikzpicture}[thick, scale = 0.8]
	\draw[fill=black] (0,0) circle (3pt);
	\draw[fill=black] (4,0) circle (3pt);
	\draw[fill=black] (2,2) circle (3pt);
	\node at (-0.5, 0) {v};
	\node at (4.5, 0) {w};
	\node at (2, 2.5) {u};
	\draw[thick] (0,0) -- (4,0);
\end{tikzpicture}
\end{center}

\begin{proof}
	首先证明必要性,假设$G = K_{n_1,\cdots, n_k} = (V, E) $是一个完全多部图,顶点可以被划分为
	\begin{equation*}
		V = \sqcup_{i=1}^k V_i.
	\end{equation*}
	则对于任意$u, v, w\in V$满足
	\begin{equation*}
		uv\notin E\text{和}uw\notin E.
	\end{equation*}
	由于$uv\notin E$于是$u,v$在同一个部分, 称作$V_i$,由$uw\notin E$,我们知道$w\in V_i$。于是有$vw\notin E$,于是得出$G$是$K_1+K_2$-free的。\par
	下面证明充分性,假设$G$是$K_1+K_2$-free的,需证$G$是完全多部图,假设$uv\notin E$,则$w$要么与$u,v$都相连要么都不相连,这定义了$G$上的一个等价关系。
\end{proof}
\begin{proposition}
	Petersen graph$P$满足下列性质:
	\begin{enumerate}
		\item $P$是三正则(cubic)的;
		\item 两个不相邻的顶点有一个共同的neighbor;
		\item $P$的girth是$5$;
		\item $P$有$10$个claw,有$10$个长为$6$的cycle。
	\end{enumerate}
\end{proposition}
\begin{definition}
	设$G$和$H$是两个图。一个$G$与$H$之间的同构映射(isomorphism)是一个双射$f:V(G)\rightarrow V(H)$满足
	\begin{equation*}
		uv\in E(G) \Leftrightarrow f(u)f(v)\in E(H)
	\end{equation*}
	即在$G$中相连的两个顶点在$H$中也相连。
\end{definition}
\begin{definition}
	令$G$是一个简单图,$G$的顶点覆盖(vertex cover)是一组与所有边都相接的顶点。顶点覆盖数(vertex cover number)是$G$中最小的$vertex cover$中的顶点个数,记做$\beta(G)$。边覆盖(edge cover)是指一组与所有边都相接的顶点个数。如果一族顶点$S$,对于任意顶点$v\in V(G)$,要么$v$在$S$中要么$v$与$S$中的某个顶点$u$相邻,则被称为$dominating set$。dominating number是指$G$中的最小的dominating set的大小,记做$\gamma(G)$。
\end{definition}
\begin{figure*}[h]
	\centering
\includegraphics[width=0.5\textwidth]{img/vertex_cover.png}
	\caption{Vertex cover\label{fig:vertex cover}}
\end{figure*}
\begin{figure*}[h]
	\centering
\includegraphics[width=0.3\textwidth]{img/dominating_set.png}
	\caption{dominating set\label{fig:dominating set}}
\end{figure*}
\begin{definition}
	如下是对图$G=(V, E)$的一些简单的操作(operation)
	\begin{enumerate}
		\item 删除$G$中的一个顶点(deleting a vertex)$v$是指从顶点集删去该点后删掉所有与该点相连的边
		\begin{equation*}
			(V\backslash \{v\}, E\backslash \{vu, u\in N(v)\})
		\end{equation*}
		\item 删除$G$中的一条边(deleting an edge)$e$
		\begin{equation*}
			(V, E\backslash \{e\})
		\end{equation*}
		\item 两张图的不交并(disjoint union) ,记做$G+H$,是指
		\begin{equation*}
			G+H = (V\sqcup V', E\sqcup E')
		\end{equation*}
		\item 两张图的并(union)记做$G\vee H$,是指
		\begin{equation*}
			G\vee H = (V\sqcup V', E\sqcup E'\cup (V\times V'))
		\end{equation*}
		是指将$G+H$的边并上所有$V\times V'$的边,即连接$V$与$V'$所有的顶点。
		\item 复制顶点(duplicating a vertex)$v$是指
		\begin{equation*}
			(V\cup\{v'\}, E\cup\{v'x:vx\in E\}).
		\end{equation*}
		$v'$被称为$v$的一个twin,意思是将$v$复制出一个$v'$,原先与$v$相连的顶点在新图中都与$v'$相连。
		\item $G$的线图(line graph),记做$L(G)$,是指图$E, F$,若边$e, e'\in E$在$G$中相邻,则在$L(G)$中该两点相连。
		\item 图$G$的subdivision是指将$G$中的某些边中加入内点,再连接起来。
		\item 一个图$H$被称作$G$的一个minor如果它能通过删除$G$边或顶点和在$G$中连线得到的。
		\item 一个图$H$被称作$G$的一个topological minor如果它包含一个是$H$的subdivision的子图。
	\end{enumerate}
\end{definition}
下面是这些操作的具体图例,
\begin{figure*}[htb]
	\centering
\includegraphics[width=0.3\textwidth]{img/disjoint_union.png}
	\caption{disjoint union of some complete graphs\label{fig:disjoint union}}
\end{figure*}
\begin{figure*}[htb]
	\centering
\includegraphics[width=0.3\textwidth]{img/LG_1.png}
	\caption{Graph G\label{fig:LG_1}}
\end{figure*}
\begin{figure*}[htb]
	\centering
\includegraphics[width=0.3\textwidth]{img/LG_2.png}
	\caption{Vertices in $L(G) $constructed from edges in $G$\label{fig:LG_2}}
\end{figure*}
\begin{figure*}[htb]
	\centering
\includegraphics[width=0.3\textwidth]{img/LG_3.png}
	\caption{Added edges in $L(G)$\label{fig:LG_3}}
\end{figure*}
\begin{figure*}[htb]
	\centering
\includegraphics[width=0.3\textwidth]{img/LG_4.png}
	\caption{The line graph $L(G)$\label{fig:LG_4}}
\end{figure*}
\begin{figure*}[htb]
	\centering
\includegraphics[width=0.3\textwidth]{img/sub_division.png}
	\caption{subdivision of $uv$\label{fig:subdivision}}
\end{figure*}
下面的定理是极值图论(extremal graph theory)
\begin{theorem}
	一个$triangle$-free的$n$顶点简单图中边数最大为$\lfloor n^2/4 \rfloor$
\end{theorem}
\begin{proof}
	设$G$是一个$n$个顶点的triangle-free简单图,令$A$是最大的独立集,其中$\alpha = |A|$。因为$G$是triangle-free的,于是每个顶点的neighborhood都是稳定的,即对于任意$v$
	\begin{equation*}
		\deg v \leq \alpha
	\end{equation*}
	。因为$V\backslash A$ 和$G$的所有顶点都相连,利用均值不等式既有
	\begin{equation*}
		|E| \leq \sum\limits_{V\in V\backslash A}\deg v\leq \alpha (n-\alpha) \leq (\frac{n}{2})^2
	\end{equation*}
\end{proof}
\begin{theorem}
	(Turan定理)设$p\geq 2$和$n\geq p+1$。令$S$是一个不包含$p$-cliques且有$n$个顶点的简单图。则所有满足这样的条件拥有最多边数的图同构与一个$p-1$部Turan图。
\end{theorem}
\begin{definition}
	一个有向图(directed graph or digraph),是一张每条边都有一个或两个方向的图,每一条边被称为一个arc。如果图中有一条从$u$到$v$的有向边,则称$v$是$u$的successor并称$u$是$v$的predecessor。进入$v$的边数称为$v$的入度(indegree),反之为出度(outdegree)。一个有向图$D$的underlying graph是指去除$D$中边的方向。一个有向图是弱连通(weakly connected)的如果它的underlying graph是连通的, 有向图$D$任意两个顶点$u,v$都有一条directed path相连则称它是强连通(strongly connected)。强连通分支(strong component)是有向图中最大的强连通子图。
\end{definition}
\begin{definition}
	竞赛图(tournament)是一个被指定方向的完全有向图。king是一个有向图中的顶点满足所有的顶点都可以至多两步到达king。
\end{definition}
\begin{proposition}
	任意一个竞赛图都有一个king。
\end{proposition}
\section{树 Trees}
\begin{theorem}
	简单图$G$是$n$阶树(Tree)如果下面等价的断言成立
	\begin{enumerate}
		\item $G$连通(connected)无圈(acyclic)。
		\item 任意两个顶点有且只有一条路径(path)。
		\item $G$是一个极大连通图,即任意删除一条边$G$都不连通。
		\item $G$是最小无圈图,即任意连一条边后$G$至少会变成有圈图。
		\item $G$连通且有$n-1$条边。
		\item $G$无圈且有$n-1$条边。
	\end{enumerate}
\end{theorem}